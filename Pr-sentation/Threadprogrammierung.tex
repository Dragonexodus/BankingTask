\documentclass{beamer} %[handout]
\usepackage[utf8x]{inputenc}
\usepackage[T1]{fontenc}
\usepackage{lmodern}
\usepackage[ngerman]{babel}
\usepackage{graphicx}
\usetheme{Berlin}
%\usecolortheme{whale}
\newcommand{\qq}[1]{\glqq{#1\grqq{}}} %Gänsefüßchen
\setbeamercovered{dynamic}
\setbeamertemplate{footline}[frame number]
\usepackage{url}
\usepackage{etoolbox}
\usepackage{listings}
\beamertemplatenavigationsymbolsempty
\appto\UrlBreaks{\do\a\do\b\do\c\do\d\do\e\do\f\do\g\do\h\do\i\do\j
\do\k\do\l\do\m\do\n\do\o\do\p\do\q\do\r\do\s\do\t\do\u\do\v\do\w
\do\x\do\y\do\z}

\lstset{
  basicstyle=\footnotesize\tt,        % the size of the fonts that are used for the code
  breakatwhitespace=false,         % sets if automatic breaks should only happen at whitespace
  breaklines=true,                 % sets automatic line breaking
  captionpos=b,                    % sets the caption-position to bottom
  extendedchars=true,              % lets you use non-ASCII characters; for 8-bits encodings only, does not work with UTF-8
  frame=single,                    % adds a frame around the code
  language=Java,                 % the language of the code
  keywordstyle=\bf,
  showspaces=false,                % show spaces everywhere adding particular underscores; it overrides 'showstringspaces'
  showstringspaces=false,          % underline spaces within strings only
  showtabs=false,                  % show tabs within strings adding particular underscores
  tabsize=2                       % sets default tabsize to 2 spaces
}


\title{Threadprogrammierung - Aufgabe 9: Bankkonten}
\subtitle{Implementierung einer Überweisung (banking.Transaction)}
\author[M. Zober, M. Horn]{Matthias Zober, Michael Horn}
\institute[HTWK - Leipzig]{Hochschule für Technik, Wirtschaft und Kultur Leipzig\\Fakultät Informatik, Mathematik und Naturwissenschaften}
\date[26. Mai 2016]{26. Mai 2016}
\begin{document}
	\maketitle	
%	\begin{frame}
%		\frametitle{Gliederung}
%		\tableofcontents
%		[pausesections]
%	\end{frame}

\section{Aufgabe}
\begin{frame}
\begin{center}
\begin{itemize}
\item Implementieren einer Überweisung von Konto 1 zu Konto 2 
\item Saldo $\ge$ 0
\item  Überweisung soll aus 2 Teiloperationen bestehen:
\begin{itemize}
\item abheben (Konto, Betrag): Verringert Saldo des Kontos um Betrag
\item einzahlen (Konto, Betrag): Erhöht Saldo des Kontos um Betrag
\end{itemize}
\end{itemize}
Jede Überweisung soll durch gleiche Sperre bewacht werden
\end{center}
\end{frame}
\section{Lösung}
\begin{frame}[fragile]
\begin{small}
\begin{lstlisting}
public void tryTransacion(Transactions) {
    l.lock();
    tries = 0;
    try {
        while (transaction.fails){
            tries++;
            c.awaitNanos(ASecond);
            if (tries > TEN) {
                c.signalAll();
                throw new RuntimeException();
            }
        }
        c.signalAll();
    } catch (InterruptedException e) 
        e.printStackTrace();
    finally
        l.unlock();
}
\end{lstlisting}

\end{small}
\end{frame}

\section{Verklemmung}
\begin{frame}


\begin{itemize}
\item Gegeben:
\begin{itemize}
	\item Salden: Konto 1 mit 5 Euro, Konto 2 mit 10 Euro
\end{itemize}
\item Überweise:
\begin{itemize}
\item 10 Euro von Konto 1 auf Konto 2
\item 15 Euro von Konto 2 auf 1 Konto
\end{itemize}
\item[$\Rightarrow$] Verklemmung: Überweisungen können niemals ausgeführt werden
\item[$\Rightarrow$] c.awaitNanos() weckt Thread nach bestimmter Zeit wieder (bekommt Lock) und erneuter Versuch der Überweisung, nach bestimmter Anzahl an Versuchen wird Exception geworfen
\end{itemize}
\end{frame}




\end{document}
